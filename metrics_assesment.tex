\section*{Discussão sobre métricas e métodos de avaliação}

Existem duas vertentes principais de métricas de erro para previsões: métricas absolutas e métricas percentuais. Cada grupo avalia o desempenho de forma diferente e é adequado para contextos distintos.

\textbf{Métricas absolutas} (MAE, MSE, RMSE) comparam previsões e valores reais na escala original da série. São estáveis quando os valores podem ser pequenos ou variar muito ao longo do tempo, pois não dependem de proporções. Têm interpretação direta (“erro médio em unidades da variável”).

\textbf{Métricas percentuais} (MAPE e sMAPE) medem erro relativo, comparando o erro ao tamanho do valor observado. São úteis quando a escala da série muda e quando o interesse é avaliar o erro como proporção do valor real. Porém, distorcem resultados quando há valores muito pequenos, já que o denominador torna o erro artificialmente grande.

Cada vertente oferece uma perspectiva diferente: métricas absolutas focam no desvio real; métricas percentuais focam na relevância relativa do erro.

Dentro do nosso contexto serão usadas métricas absolutas que preservam a escala original dos dados, permitindo comparar diretamente as diferenças entre valores observados e previstos. Essa escolha facilita a interpretação dos resultados, mantendo a coerência com a unidade de medida da variável analisada. Essas métricas permanecem adequadas mesmo com covariáveis, pois não dependem da escala relativa dos valores observados.

Além disso, existem diferentes formas de treinar o modelo para previsões em séries temporais. Os principais tipos são:

\textbf{One-shot}: Gera todos os passos futuros de uma vez. A entrada é a janela histórica, a saída é um vetor contendo todos os pontos de previsão. Essa método evita acúmulo de erro, porém mais difícil de treinar quando o horizonte é grande.

\textbf{Iterativo}: Prevê um passo por vez. Para prever o passo seguinte, usa como entrada a própria previsão anterior. Gera $\hat{y}(t+1)$, depois usa $\hat{y}(t+1)$ para gerar $\hat{y}(t+2)$, e assim por diante. Uma desvantagem dessa abordagem é o erro que se acumula a cada passo.

\textbf{Lag}: Transforma a série em um problema supervisionado. O histórico recente vira feature e o próximo valor vira alvo. Um ponto é que é flexível e compatível com qualquer modelo, porém é sensível a escolha dos lags. Com covariáveis, é possível usar valores no mesmo instante da variável-alvo (contemporâneos), conhecidas no presente, e também versões defasadas dessas covariáveis, quando só é conhecido o passado e não o presente.