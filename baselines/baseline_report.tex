\section*{Modelos Baseline}

Modelos baseline são abordagens de previsão simples que servem como um ponto de referência para avaliar o desempenho de modelos mais complexos. A sua principal função é estabelecer um limiar de performance: se um modelo sofisticado não conseguir superar um baseline simples, isso sugere que o modelo mais complexo não está capturando bem os padrões da série ou que a série é inerentemente difícil de prever. Para este trabalho, foram implementados cinco modelos baseline, cujos resultados foram comparados utilizando as métricas MAE, RMSE e MAPE.

\subsection*{Descrição dos Modelos}

\textbf{Mean Method:} Este é o baseline mais simples, onde a previsão para qualquer período futuro é a média de todos os valores observados no conjunto de treino.
$$ \hat{y}_{T+h|T} = \bar{y} = \frac{1}{T}\sum_{t=1}^{T}y_t $$

\textbf{Naive Method:} Neste método, a previsão para o próximo período é simplesmente o último valor observado.
$$ \hat{y}_{T+h|T} = y_T $$

\textbf{Seasonal Naive:} Uma variação do Naive Method, útil para séries com forte sazonalidade. Ele prevê o valor futuro utilizando a observação do mesmo período no ciclo sazonal anterior. Utilizamos dados semanais considerando sazonalidade anual, portanto o período é $m=52$.
$$ \hat{y}_{T+h|T} = y_{T+h-m} $$

\textbf{Drift Method:} Este método é uma extensão do Naive e é adequado para séries com tendência. Ele extrapola uma linha traçada entre a primeira e a última observação do conjunto de treino. A previsão incorpora uma ``inclinação'' (drift) média.
$$ \hat{y}_{T+h|T} = y_T + h \left( \frac{y_T - y_1}{T-1} \right) $$

\textbf{Média Móvel (Rolling Mean):} A previsão é calculada como a média dos últimos $K$ valores observados no conjunto de treino. Este método suaviza flutuações de curto prazo. Para este estudo, foi utilizado $K=12$.
$$ \hat{y}_{T+h|T} = \frac{1}{K}\sum_{t=T-K+1}^{T}y_t $$

\subsection*{Metodologia de Validação}

Para avaliar o desempenho dos modelos de forma robusta, foi empregada a técnica de \textit{Time Series Cross-Validation} com janela expansível. Esta abordagem consiste em:

\begin{itemize}
    \item \textbf{Janela inicial de treino:} 120 observações
    \item \textbf{Horizonte de previsão:} 30 pontos por fold
    \item \textbf{Passo de deslocamento:} 1 observação entre folds consecutivos
    \item \textbf{Número de folds:} 8 folds
\end{itemize}

Em cada fold, o modelo é treinado com os dados até o ponto de corte e realiza previsões \textit{one-step-ahead rolling forecast} para o horizonte de 30 pontos. Nesta estratégia, ao prever o ponto $h+1$, utilizamos todos os dados observados até o ponto $h$ (incluindo os valores reais do teste até esse ponto), garantindo uma avaliação realista da capacidade preditiva dos modelos.

A Figura \ref{fig:baselines_comparison} ilustra o comportamento de todos os modelos baseline no último fold, permitindo visualizar como cada método se comporta em relação aos valores observados.

\begin{figure}[H]
    \centering
    \includegraphics[width=1\textwidth]{baselines_comparison.png}
    \caption{Comparação visual de todos os modelos baseline no último fold de validação. A linha preta representa os dados de treino, a linha vermelha os valores reais do teste, e as linhas tracejadas coloridas representam as previsões de cada modelo baseline.}
    \label{fig:baselines_comparison}
\end{figure}

\subsection*{Resultados}

Os resultados obtidos para as métricas de erro, calculados como a média sobre os 8 folds de validação cruzada, estão consolidados na Tabela \ref{tab:baseline_results}. O \textbf{Drift Method} apresentou o melhor desempenho geral, seguido de perto pelo Naive Method.

\begin{table}[h]
    \centering
    \begin{tabular}{|l|c|c|c|}
        \hline
        \textbf{Modelo} & \textbf{MAE} & \textbf{RMSE} & \textbf{MAPE (\%)} \\
        \hline
        Drift Method & 1.6139 & 2.1241 & 13.34 \\
        Naive Method & 1.6258 & 2.1323 & 13.39 \\
        Rolling Mean (K=12) & 2.3230 & 3.0989 & 17.68 \\
        Seasonal Naive & 6.0535 & 7.3639 & 46.48 \\
        Mean Method & 8.4744 & 9.1184 & 69.71 \\
        \hline
    \end{tabular}
    \caption{Resultados médios dos modelos baseline na validação cruzada (8 folds). Modelos ordenados por RMSE crescente (melhor para pior).}
    \label{tab:baseline_results}
\end{table}

O modelo com melhor desempenho (\textbf{Drift Method}) foi selecionado para uma análise de resíduos detalhada, apresentada na próxima subseção.

Para aplicações práticas, saber a margem de erro é tão importante quanto a previsão em si. Avaliamos os intervalos de confiança (IC) de 95\% gerados pelo Drift Method utilizando métricas probabilísticas robustas. A Figura \ref{fig:drift_ci} apresenta a previsão visual juntamente com a faixa de incerteza estimada.

\begin{figure}[H]
    \centering
    \includegraphics[width=1\textwidth]{Driftconfidence.png}
    \caption{Previsão \textit{one-step-ahead} do Drift Method no último fold, exibindo o Intervalo de Confiança de 95\% (área sombreada). A caixa de texto destaca as métricas de calibração do intervalo.}
    \label{fig:drift_ci}
\end{figure}

As métricas obtidas demonstram que o estimador de incerteza do Drift Method é confiável:

\begin{itemize}
    \item \textbf{Cobertura (Coverage):} O modelo obteve \textbf{93.33\%} de cobertura, valor muito próximo do ideal teórico de 95\%. Isso significa que, em 93\% das semanas, o valor real de volume caiu dentro da faixa prevista, indicando um modelo bem calibrado (nem excessivamente confiante, nem conservador demais).
    
    \item \textbf{Winkler Score (10.69):} Este score penaliza intervalos largos que não capturam o valor real. O valor baixo obtido, em conjunto com a alta cobertura, sugere que o modelo consegue manter o intervalo "estreito o suficiente" para ser útil (Largura Média $\approx$ 8.53) sem sacrificar a precisão.
    
    \item \textbf{Violação (6.67\%):} A taxa de violação está marginalmente acima dos 5\% esperados, o que é aceitável para um modelo baseline.
\end{itemize}

Esta validação da incerteza reforça a escolha do Drift Method como um benchmark sólido: ele não apenas prevê a tendência corretamente, mas também fornece uma estimativa de risco realista para o negócio.

\subsection*{Análise de Resíduos do Melhor Modelo (Drift)}

A análise de resíduos do melhor modelo baseline foi realizada em três etapas:

\textbf{1. Série Temporal dos Resíduos:} A série temporal dos resíduos parece apresentar algum padrão temporal, podemos ver alguns intervalos que se repetem em direção e diferem pouco em amplitude. A maioria dos resíduos concentrados em torno de zero. No entanto, a variância pareceu aumentar ligeiramente em alguns períodos, sugerindo possível heteroscedasticidade.

\begin{figure}[H]
    \centering
    \includegraphics[width=1\linewidth]{residuo.png}
    \caption{Série Temporal dos Resíduos}
    \label{fig:placeholder}
\end{figure}
\textbf{2. Normalidade dos Resíduos:} A análise de normalidade através do histograma e gráfico Q-Q revelou desvios em relação à distribuição normal, especialmente nas caudas. Testes estatísticos confirmaram a rejeição da hipótese de normalidade, indicando que, embora o modelo capture a tendência central (média), ele pode não capturar perfeitamente os eventos extremos.
\begin{figure}[H]
    \centering
    \includegraphics[width=1\linewidth]{normalidade.png}
    \caption{Normalidade dos Resíduos}
    \label{fig:placeholder}
\end{figure}
\textbf{3. Autocorrelação dos Resíduos:} A análise de autocorrelação (ACF) mostrou correlação serial significativa em alguns lags iniciais. Isso era esperado para um modelo baseline simples e indica que há estrutura temporal de curto prazo remanescente que não foi capturada apenas pela tendência linear do Drift. \begin{figure}[H]
    \centering
    \includegraphics[width=1\linewidth]{autocorrelacao.png}
    \caption{Autocorrelação dos Resíduos}
    \label{fig:placeholder}
\end{figure}
\subsection*{Transformação Box-Cox e Heteroscedasticidade}

A análise inicial dos dados sugeriu heteroscedasticidade. Para mitigar este problema, aplicou-se a transformação de Box-Cox. 

Após aplicar a transformação, os modelos foram reavaliados. Os resultados mostraram diferenças marginais nas métricas de erro. Por exemplo, o MAE do Drift Method variou apenas de \textbf{1.614 para 1.608} com a transformação.

Esta pequena variação pode ser explicada por dois fatores principais:

\begin{itemize}
    \item \textbf{Dominância da Tendência:} A série possui uma tendência linear forte que é bem capturada pelo Drift Method independentemente da escala de variância.
    \item \textbf{Horizonte Curto:} Com previsão \textit{one-step-ahead}, o modelo é re-calibrado a cada passo, reduzindo o impacto negativo da heteroscedasticidade que seria mais visível em horizontes longos.
\end{itemize}

\subsection*{Conclusão}

Os modelos baseline estabelecem um ponto de referência importante. O \textbf{Drift Method} (MAE $\approx 1.61$) provou ser um benchmark robusto, superando significativamente abordagens sazonais ou de média simples. A análise de resíduos evidenciou que, embora a tendência tenha sido removida, ainda existe autocorrelação a ser explorada por modelos estatísticos mais sofisticados.
